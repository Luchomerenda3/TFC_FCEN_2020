\documentclass[a4paper,12pt]{article}
\usepackage[spanish]{babel}
\usepackage{avant}
\usepackage[utf8]{inputenc}
\usepackage{graphicx}
\usepackage{booktabs}
\usepackage{amsmath}
\usepackage{amsfonts}

\begin{document}
\title{Taller de Física Computacional 2020}

\author{Primer Parcial}

\maketitle

\noindent {\bf IMPORTANTE:} 
\begin{itemize}
    \item Resuelva los ejercicios en un notebook nuevo que deberá subir como respuesta a esta tarea una vez resuelto.
    \item Comente el código de manera que sea legible por otra persona.
    \item Todas las celdas deben ejecutar sin error. Se recomienda hacer un reinicio del núcleo y evaluar todas las celdas en orden para asegurarse que ejecuta como corresponde antes de entregar.
    \item Lea {\bf atentamente} las consignas.
\end{itemize}

\begin{enumerate}
    \item[\bf Ejercicio 1] ({\em 20 puntos}) Graficar la función 
    $$ f(x) = e^{-x}\sin 10x $$
    en el intervalo $[0,10]$.
    \item[\bf Ejercicio 2] ({\em 20 puntos}) Implementar una función que clasifique un triángulo como
    rectángulo, acutángulo u obtusángulo dadas las longitudes de sus lados $a$, $b$ y $c$ (números reales)
    en base a las condiciones enumeradas en la siguiente tabla:
    \begin{center}
        \begin{tabular}{@{}ll@{}}
            Condición            & Tipo                                               \\ \midrule
            si $a^2 + b^2 = c^2$ & el triángulo es {\bf rectáncgulo} \\ 
            si $a^2 + b^2 > c^2$ & el triángulo es {\bf acutángulo}  \\
            si $a^2 + b^2 < c^2$ & el triángulo es {\bf obtusángulo} \\ \bottomrule
            \end{tabular}  
    \end{center}
    \item[\bf Ejercicio 3] ({\em 20 puntos}) Implementar una función que determine las
    componentes del vector $\mathbf{C}$, donde $\mathbf{C}=\mathbf{A}\times\mathbf{B}$
    es el producto vectorial entre los vectores $\mathbf{A}$ y $\mathbf{B}$. 
    Nota: usar {\tt numpy.cross} o similar es {\bf trampa}.
    \item[\bf Ejercicio 4]  ({\em 20 puntos}) La ecuación Pitagórica 
    $$i^2+j^2=k^2$$
    tiene un número infinito de soluciones enteras {\bf positivas} llamadas 
    ``triplas Pitagóricas''. 
    \begin{enumerate}
        \item[a] ({\em 10 puntos}) Diseñar e implementar una función que calcule (imprima) todas las
        triplas Pitagóricas para un dado $m$ tal que $i \le m$, $j\le m$ y $k\le m$.
        No importa si el código devuelve algunas triplas repetidas.
        \item[b] ({\em 10 puntos}) Modificar la función para ``probar'' que el último teorema de Fermat, que
        asegura que no existe ninguna solución de enteros positivos a $$i^n+j^n=k^n,$$
        es válido hasta un cierto entero positivo $m$.
        \item[c] ({\bf opcional}) Modificar la función del ítem (a) para que devuelva
        una secuencia de tuplas que contenga las triplas únicas.
    \end{enumerate}
    \item[Ejercicio 5] ({\em 20 puntos}) Una serie de Fourier tiene la forma
    $$s(x) = \frac{a_0}{2}+\sum_{i=1}^N \left( a_n \cos(nx) + b_n \sin(nx) \right)$$
    y puede aproximar (con valores adecuados de $\{a_n\}$ y $\{b_n\}$) cualquier función
    periódica de período $2\pi$.
    \begin{enumerate}
        \item ({\em 8 puntos}) Implementar una función que acepte como parámetros dos listas conteniendo los conjuntos de coeficientes $\{a_n\}$ y $\{b_n\}$ y evalúe la correspondiente serie de Fourier para un argumento $x\in \mathbb{R}$.
        \item ({\em 8 puntos}) Utilizar la función anterior para implementar una nueva función que evalúe la serie de Fourier de la función {\em serrucho} para la
        cual 
        $$a_n = 0\ \forall\ n$$
        y
        $$b_n = \frac{2(-1)^{n+1}}{n \pi}\ \forall\ n\ge 1$$
        \item ({\em 4 puntos}) Comparar gráficamente la aproximación de órden $10$ para la función serrucho. 
        Utilizar la siguiente definición para la función serrucho:
        \begin{verbatim}
def f(x):
    return 2*((x+1/2) - math.floor(x+1/2)) - 1   
        \end{verbatim}
        Nota: En esta definición $f(x)$ es periódica con período 1.
    \end{enumerate}
\end{enumerate}

\end{document}
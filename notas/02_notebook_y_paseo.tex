\documentclass{beamer}

\usepackage[utf8]{inputenc}
\usepackage[spanish]{babel}
\usepackage[outputdir=.build]{minted}
\usepackage{hyperref}
\usepackage{graphicx}

\usetheme{Madrid}

%Information to be included in the title page:
\title{Taller de Física Computacional}
\subtitle{Plataforma {\em Jupyter Notebook} y un paseo por Python}
\author{Cristián G. Sánchez y Carlos J. Ruestes}
\date{2020}

\begin{document}

\frame{\titlepage}

%%%%%%%%%%%%%%%%%%%%%%%%%%%%%%%%%%%%%%%%%%%%%%%%%%%%%%%%%%
\begin{frame}
\begin{block}{\em Jupyter Notebooks}
\begin{itemize}
\item Hay muchas formas de utilizar Python.
\item El hecho de que sea un lenguaje interpretado es la razón de esta multiplicidad.
\item Para este curso vamos a utilizar la herramienta {\em Jupyter Notebooks}
\end{itemize}
\end{block}
\end{frame}

%%%%%%%%%%%%%%%%%%%%%%%%%%%%%%%%%%%%%%%%%%%%%%%%%%%%%%%%%%
\begin{frame}
\frametitle{¿Qué es Jupyter Notebook?}
\begin{itemize}
\item {\em Jupyter notebook es una plataforma de {\em programación interactiva}}. 
\item Es muy útil en contextos de enseñanza (como este) 
\item (Algunos) programadores sazonados lo utilizan para construir prototipos y explorar datos.
\item Tiene fuertes detractores en la comunidad Python.
\item Tiene un modelo de ejecución \alert{no lineal}: El orden en el que las cosas están escritas no necesariamente es el orden en que se ejecutan.
\end{itemize}
\end{frame}

%%%%%%%%%%%%%%%%%%%%%%%%%%%%%%%%%%%%%%%%%%%%%%%%%%%%%%%%%%%
\begin{frame}[plain,c]

Pasemos a una demostración.

\end{frame}

%%%%%%%%%%%%%%%%%%%%%%%%%%%%%%%%%%%%%%%%%%%%%%%%%%%%%%%%%%%
\begin{frame}
\frametitle{Tarea}

\begin{itemize}
\item Toda la práctica se desarrollará resolviendo problemas utilizando la interfaz Jupyter Notebook para Python.
\item Google Colaboratory va a ser la plataforma para la parte práctica del curso \url{https://colab.research.google.com/}.
\item Pueden usar Jupyter Notebooks localmente u otros sitios como Binder. Instrucciones del 2019 en \url{https://github.com/cjruestes/TFC_FCEN_2019/blob/master/doc/Instrucciones_para_ejecutar_notebooks.md} 
\end{itemize}
\end{frame}




\end{document}

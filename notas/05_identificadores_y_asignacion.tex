\documentclass{beamer}

\usepackage[utf8]{inputenc}
\usepackage[spanish]{babel}
\usepackage[outputdir=.build]{minted}
\usepackage{hyperref}
\usepackage{graphicx}
\usepackage{amsmath}
\usepackage{calrsfs}


\usepackage{tikz}
\usetikzlibrary{positioning,arrows}

\hypersetup{
    colorlinks = true
}

\usetheme{Madrid}

%Information to be included in the title page:
\title{Taller de Física Computacional}
\subtitle{Identificadores y asignación}
\author{Cristián G. Sánchez y Carlos J. Ruestes}
\date{2020}

\begin{document}

\frame{\titlepage}

%%%%%%%%%%%%%%%%%%%%%%%%%%%%%%%%%%%%%%%%%%%%%%%%%%%%%%%%%%
\begin{frame}[fragile]
\frametitle{Identificadores o nombres}
\begin{block}{{\em Identificadores o nombres}}
\begin{itemize}
\item Los identificadores son ``palabras'' armadas con un subconjunto de ``letras'' o caracteres que no sean espacios.
\item Las reglas para los posibles identificadores son muchas, particularmente con caracteres Unicode.
\item Algunos ejemplos de identificadores válidos son: \mintinline{python}{a0, AceleraciónLineal, number_02}
\end{itemize}
\end{block}
\end{frame}

%%%%%%%%%%%%%%%%%%%%%%%%%%%%%%%%%%%%%%%%%%%%%%%%%%%%%%%%%%
\begin{frame}[fragile]
\frametitle{Identificadores o nombres}
\begin{block}{{\em Identificadores o nombre}}
\begin{itemize}
\item Los identificadores o nombres sirven para ponerle nombre a las cosas (!)
\item Cuando decimos ``cosas'' estamos hablando de {\em objetos}.
\item Hay muchas formas en que un objeto recibe un nombre.
\item Por ahora nos interesan las \alert{asignaciones}: {\tt nombre = algo}
\end{itemize}
\end{block}
\end{frame}

%%%%%%%%%%%%%%%%%%%%%%%%%%%%%%%%%%%%%%%%%%%%%%%%%%%%%%%%%%
\begin{frame}[fragile]
\frametitle{Asignación}
\begin{block}{Asignación}
Una asignación tiene la estructura \mintinline{python}{nombre = objeto} o  \mintinline{python}{nombre = relación u operación entre objetos}.
\begin{minted}{python}
a = 2.0**(1/2)  + 1
b = a - 3
c = "hoy hay sol"
d = f"b vale {b}"
\end{minted}
\end{block}
\end{frame}

%%%%%%%%%%%%%%%%%%%%%%%%%%%%%%%%%%%%%%%%%%%%%%%%%%%%%%%%%%
\begin{frame}[fragile]
\frametitle{Asignación}
\begin{block}{Asignación múltiple}
Las siguientes asignaciones son válidas:
\begin{minted}{python}
a, b = 1, 2
c, d = b + 1, b + 2
a, b = b, a
\end{minted}
En una asignación se evalúan primero las operaciones a la derecha de la asignación y las asignaciones se hacen de izquierda a derecha en la lista.
(\dag) Las asignaciones múltiples hacen uso de \alert{tuplas}, un tipo que aún no hemos visto.
\end{block}
\end{frame}

%%%%%%%%%%%%%%%%%%%%%%%%%%%%%%%%%%%%%%%%%%%%%%%%%%%%%%%%%%
\begin{frame}[fragile]
\frametitle{Asignación aumentada} 
\begin{block}{Asignación aumentada}
Si \mintinline{python}{a} y \mintinline{python}{b} representan números (\dag) la siguiente asignación \mintinline{python}{a += b} 
\begin{minted}{python}
a = 1
a += 1 # es lo mismo que a = a + 1
\end{minted}
\end{block}
(\dag) números u objetos para los cuales la operación binaria \mintinline{python}{+} esté definida
\end{frame}

%%%%%%%%%%%%%%%%%%%%%%%%%%%%%%%%%%%%%%%%%%%%%%%%%%%%%%%%%%
\begin{frame}[fragile]
\frametitle{En la práctica} 
\begin{block}{En la práctica}
Los nombres se utilizan para guardar resultados intermedios que se van combinando o transformando para generar el resultado final (\dag).
En general a los nombres les vamos a llamar \alert{variables}.
\end{block}
(\dag) Si es que existe\ldots
\end{frame}

%%%%%%%%%%%%%%%%%%%%%%%%%%%%%%%%%%%%%%%%%%%%%%%%%%%%%%%%%%
\begin{frame}[fragile]
\frametitle{Programa y estado} 
\begin{block}{Programa}
Un programa es una secuencia de instrucciones que modifican el \alert{estado}.
\end{block}
\begin{block}{Estado}
El estado de un programa en un determinado momento está representado por el conjunto de objetos que tienen nombre (variables), es decir, son accesibles.
\end{block}
\end{frame}

%%%%%%%%%%%%%%%%%%%%%%%%%%%%%%%%%%%%%%%%%%%%%%%%%%%%%%%%%%
\begin{frame}[fragile]
\frametitle{Programa y estado} 
%Diagram
\begin{figure}
\begin{tikzpicture}[node distance=2cm]

\tikzstyle{squarednode} = [rectangle, draw=blue!60, fill=blue!5, very thick, minimum height=10mm, minimum width=40mm, text centered]
\tikzstyle{darrow} = [very thick,dashed,->,>=stealth]
\tikzstyle{arrow} = [very thick,->,>=stealth]

%Nodes
\node (E0) [squarednode] 			     {$\mathcal{E}_0 = \lbrace a_0,b_0,c_0,\ldots \rbrace$};
\node (E1) [squarednode, below of=E0] {$\mathcal{E}_0 = \lbrace a_0,b_0,c_0,\ldots \rbrace$};
\node (E2) [squarednode, below of=E1] {$\mathcal{E}_n = \lbrace a_n,b_n,c_0,\ldots \rbrace$};
\node (Ef)  [squarednode, below of=E2] {$\mathcal{E}_f = \lbrace a_f,b_f,c_f,\ldots \rbrace$};

\draw [arrow] (E0) -- node[anchor=west] {$\mathcal{P}$} (E1);
\draw [darrow] (E1) -- node[anchor=west] {$\mathcal{P}$} (E2);
\draw [darrow] (E2) -- node[anchor=west] {$\mathcal{P}$} (Ef);

\end{tikzpicture}
\end{figure}
\end{frame}


%%%%%%%%%%%%%%%%%%%%%%%%%%%%%%%%%%%%%%%%%%%%%%%%%%%%%%%%%%%
\begin{frame}
\frametitle{Síntesis y recursos:}

\begin{itemize}
\item \href{https://docs.python.org/3/reference/index.html}{Manual de referencia de Python}
\item \href{https://docs.python.org/3/library/index.html}{Manual de la Librería estándar de Python}
\end{itemize}
\end{frame}

\end{document}

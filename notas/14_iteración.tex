\documentclass{beamer}

\usepackage[utf8]{inputenc}
\usepackage[spanish]{babel}
\usepackage[outputdir=.build]{minted}
\usepackage{hyperref}
\usepackage{graphicx}

\hypersetup{
    colorlinks = true
}

\usetheme{Madrid}

%Information to be included in the title page:
\title{Taller de Física Computacional}
\subtitle{Iteración sobre secuencias}
\author{Cristián G. Sánchez y Carlos J. Ruestes}
\date{2020}

\begin{document}

\frame{\titlepage}

%%%%%%%%%%%%%%%%%%%%%%%%%%%%%%%%%%%%%%%%%%%%%%%%%%%%%%%%%%
\begin{frame}[fragile]
    \frametitle{Iteración sobre secuencias}
    \begin{block}{{\em Iteración sobre secuencias}}
    Python provee la declaración \mintinline{python}{for} para iterar sobre secuencias:
    \begin{minted}[showspaces=true,autogobble=true]{python}
        for x in s:
            # esto se repite tantas veces como elementos
            # tenga s y x toma el valor i-esimo en la iteración i
    \end{minted}
    Dada una secuencia, por ejemplo una lista \mintinline{python}{s = [1,2,3,4,5]}, el bloque que sigue a la línea contiendo la declaración \mintinline{python}{for}
    se ejecutará 5 veces. En cada iteración la variable \mintinline{python}{x} tomará los valores 1,2,3,4 y 5. 
    \end{block}
\end{frame}

%%%%%%%%%%%%%%%%%%%%%%%%%%%%%%%%%%%%%%%%%%%%%%%%%%%%%%%%%%
\begin{frame}[fragile]
    \frametitle{Iteración sobre un rango}
    \begin{block}{{\em Iteración sobre un rango}}
    Los rangos se utilizan en muchos casos para representar una secuencia de enteros sobre la cual iterar sin necesariamente construir una lista que contenga los valores:
    \begin{minted}[showspaces=true,autogobble=true]{python}
        for i in range(0,n):
            # esto se repite n veces, i toma los valores desde
            # 0 a (n-1)
    \end{minted}
    Esta es, por lejos, la estructura de bucle más utilizada en Python y representa una alternativa a la declaración \mintinline{python}{while}.
    \end{block}
    Las palabras clave \mintinline{python}{break} y \mintinline{python}{continue} se pueden utilizar de la misma manera que en el bucle \mintinline{python}{while}.
    \mintinline{python}{break} sale incondicionalmente y \mintinline{python}{continue} hace que i tome el siguente valor y el bloque se ejecute desde el inicio salteando lo
    que siga.
\end{frame}

%%%%%%%%%%%%%%%%%%%%%%%%%%%%%%%%%%%%%%%%%%%%%%%%%%%%%%%%%%
\begin{frame}[fragile]
    \frametitle{Iteración sobre un \mintinline{python}{enumerate}}
    \includegraphics[height=0.8cm, width=0.8cm]{figuras/sin.png}
    \begin{block}{{\em Iteración sobre un \mintinline{python}{enumerate}}}
    La declaración \mintinline{python}{for} se puede utilizar con objetos {\em iterables} o funciones especiales llamadas {\em generadores}. Si bien no vamos a entrar en esa
    rama hay un generador que es muy útil: \mintinline{python}{enumerate(s)} devuelve, en cada iteración, una tupla \mintinline{python}{(i,x)} que contiene el valor y su índice:
    \begin{minted}[showspaces=true,autogobble=true]{python}
        for i,x in enumerate(s):
            # esto se repite tantas veces como elementos
            # tenga s y x e i toman el valor i-ésimo y su índice, 
            # respectivamente, en cada iteración.
    \end{minted}
    \end{block}
\end{frame}

%%%%%%%%%%%%%%%%%%%%%%%%%%%%%%%%%%%%%%%%%%%%%%%%%%%%%%%%%%
\begin{frame}[fragile]
    \frametitle{Ejemplo}
    \begin{block}{Evaluar una función en un intervalo de la recta real}
    Podemos modificar el ejemplo que habíamos utilizado en la demo de bucles para, ahora que sabemos como, almacenar una tabla de valores
    en vez de mostrarla en pantalla:
    \begin{minted}{python}
    n = 1000; xmin = -1.0; xmax = 1.0
    xs = []; ys =[]

    for i in range(0,n+1):
        x = xmin + i * (xmax - xmin) / n
        xs += [x]
        ys += [f(x)]

    \end{minted}
    Ahora la lista \mintinline{python}{xs} contiene la grilla sobre la que evaluamos a la función y la lista \mintinline{python}{ys} el valor
    en cada punto. Notar que usamos \mintinline{python}{(n+1)} en el rango, \mintinline{python}{n} es la cantidad de sub-intervalos en la grilla
    que contiene \mintinline{python}{(n+1)} puntos.
    \end{block}
    \end{frame}

%%%%%%%%%%%%%%%%%%%%%%%%%%%%%%%%%%%%%%%%%%%%%%%%%%%%%%%%%%
\begin{frame}[fragile]
    \frametitle{Comprehensiones}
    \includegraphics[height=0.8cm, width=0.8cm]{figuras/sin.png}
    \begin{block}{Comprehensiones}
    Las {\em comprehensiones} proveen una forma concisa de crear listas. Las comprhensiones utilizan
    una o más palabras clave \mintinline{python}{for} o \mintinline{python}{if}
    \begin{minted}[autogobble=true]{python}
        # lista de numeros entre 0 y 99
        s = [i for i in range(0,100)]

        # números pares entre 0 y 99
        s = [i for i in range(0,100) if i%2 ==0]

        # pueden anidarse
        s = [[i*j for i in range(0,100)] for j in range(0,100)]
    \end{minted}
    En estos ejemplos hemos usado rangos pero se puede utilizar cualquier iterable.
    \end{block}

\end{frame}

%%%%%%%%%%%%%%%%%%%%%%%%%%%%%%%%%%%%%%%%%%%%%%%%%%%%%%%%%%
\begin{frame}[fragile]
    \includegraphics[height=0.8cm, width=0.8cm]{figuras/sin.png}
    \frametitle{Mapas}
    \begin{block}{Mapas}
    Es posible construir un generador que represente la aplicación de una función a cada miembro de una lista
    utilizando la función \mintinline{python}{map()}
    \begin{minted}[autogobble=true]{python}
        # lista de numeros entre 0 y 99
        xs = [-1 + 2 * i / 100 for i in range(0,101)]

        # una función
        def f(x):
            return np.sin(x)/(x+3)

        # lista de valores de f sobre xs
        ys = list(map(f,xs))
    \end{minted}
    En el último ejemplo convertimos el iterable en una lista.
    \end{block}

\end{frame}

%%%%%%%%%%%%%%%%%%%%%%%%%%%%%%%%%%%%%%%%%%%%%%%%%%%%%%%%%%
\begin{frame}[fragile]
    \includegraphics[height=0.8cm, width=0.8cm]{figuras/sin.png}
    \frametitle{Funciones anónimas}
    \begin{block}{Funciones anónimas, expresiones \mintinline{python}{lambda}}
    Para evitar definir la función en el ejemplo anterior podemos utilizar la expresión
    \mintinline{python}{lambda} que nos permite definir una función {\em anónima} en
    el lugar en el que la necesitamos.
    \begin{minted}[autogobble=true]{python}
        # lista de numeros entre 0 y 99
        xs = [-1 + 2 * i / 100 for i in range(0,101)]
        # lista de valores de f sobre xs
        ys = list(map(lambda x : np.sin(x)/(x+3),xs))
    \end{minted}
    Las expresiones \mintinline{python}{lambda} pueden usarse en cualquier lugar que debamos pasar 
    una función como argumento y permiten usar la función sin definirla explícitamente.
    \end{block}

\end{frame}

%%%%%%%%%%%%%%%%%%%%%%%%%%%%%%%%%%%%%%%%%%%%%%%%%%%%%%%%%%%
\begin{frame}
\frametitle{Síntesis y recursos:}

\begin{itemize}
    \item \href{https://docs.python.org/3/reference/index.html}{Manual de referencia de Python}
    \item \href{https://docs.python.org/3/library/index.html}{Manual de la Librería estándar de Python}
\end{itemize}
\end{frame}

\end{document}

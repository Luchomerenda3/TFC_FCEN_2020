\documentclass{beamer}

\usepackage[utf8]{inputenc}
\usepackage[spanish]{babel}
\usepackage[outputdir=.build]{minted}
\usepackage{hyperref}
\usepackage{graphicx}

\hypersetup{
    colorlinks = true
}

\usetheme{Madrid}

%Information to be included in the title page:
\title{Taller de Física Computacional}
\subtitle{Operaciones Lógicas}
\author{Cristián G. Sánchez y Carlos J. Ruestes}
\date{2020}

\begin{document}

\frame{\titlepage}

%%%%%%%%%%%%%%%%%%%%%%%%%%%%%%%%%%%%%%%%%%%%%%%%%%%%%%%%%%
\begin{frame}[fragile]
    \frametitle{Módulos}
    \begin{block}{{\em Módulos}}
    De acuerdo al \href{https://docs.python.org/3/glossary.html}{Glosario de Python}:
    \begin{quote}
    Un {\em módulo} es un objeto que sirve como una unidad organizacional de código Python. Los módulos tienen un {\em espacio de nombres} que contienen objetos de Python arbitrarios. Los módulos se cargan en Python a través del proceso de {\em importación}.
    \end{quote}
    \end{block}
    \begin{itemize}
    \item En la práctica los módulos permiten construir bibliotecas de funciones, clases, variables con valores, etc..
    \item Los módulos se escriben en un archivo {\tt nombre.py} que debe residir en una carpeta accesible a Python.
    \item Los módulos pueden organizarse en paquetes que pueden a su vez contener otros sub-módulos y/o sub-paquetes.
    \end{itemize}
    
    \end{frame}

%%%%%%%%%%%%%%%%%%%%%%%%%%%%%%%%%%%%%%%%%%%%%%%%%%%%%%%%%%
\begin{frame}[fragile]
    \frametitle{Sintaxis}
    \begin{block}{Sintaxis}
    Para importar un módulo o un módulo parte de un paquete con todos sus métodos pueden usarse las sintaxis:
    \begin{minted}{python}
    import modulo 
    from paquete import módulo
    \end{minted}
    o
    \begin{minted}{python}
    import modulo as alias
    from paquete import módulo as alias
    \end{minted}
    En el caso de usar un alias este reemplaza el nombre del módulo para llamar a sus métodos.
    \end{block}
    \end{frame}

%%%%%%%%%%%%%%%%%%%%%%%%%%%%%%%%%%%%%%%%%%%%%%%%%%%%%%%%%%%
\begin{frame}
\frametitle{Síntesis y recursos:}

\begin{itemize}
\item \href{https://numpy.org/doc/stable/}{Documentación de NumPy}
\item \href{https://numpy.org/doc/stable/reference/routines.math.html}{Funciones matemáticas en NumPy}

\end{itemize}
\end{frame}

\end{document}

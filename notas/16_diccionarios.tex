\documentclass{beamer}

\usepackage[utf8]{inputenc}
\usepackage[spanish]{babel}
\usepackage[outputdir=.build]{minted}
\usepackage{hyperref}
\usepackage{graphicx}

\hypersetup{
    colorlinks = true
}

\usetheme{Madrid}

%Information to be included in the title page:
\title{Taller de Física Computacional}
\subtitle{Diccionarios}
\author{Cristián G. Sánchez y Carlos J. Ruestes}
\date{2020}

\begin{document}

\frame{\titlepage}

%%%%%%%%%%%%%%%%%%%%%%%%%%%%%%%%%%%%%%%%%%%%%%%%%%%%%%%%%%
\begin{frame}[fragile]
    \frametitle{Diccionarios}
    \begin{block}{Diccionarios}
        Los diccionarios son listas asociativas en las que el índice puede ser cualquier objeto {\em hasheable}.
        La razón de este requisito es que que se implementan como {\em hashtables}. 
        \begin{itemize}
        \item En este curso no vamos a profundizar sobre los diccionarios.
        \item Se usar como parámetros para algunas funciones de SciPy y por lo tanto es necesario conocerlos para usarlas.
        \item Si bien en comptación científica las estructuras de datos más utilizadas son los {\em arreglos multidimensionales}, de los cuales los vectores, matrices y tensores son ejemplos,
        las {\em hashtables} pueden tener aplicaciones inesperadas. Un ejemplo es agilizar el acceso a tensores de muchísimas dimensiones a través de un {\em hash} de sus índices.
        \end{itemize}
    \end{block}
    \end{frame}

%%%%%%%%%%%%%%%%%%%%%%%%%%%%%%%%%%%%%%%%%%%%%%%%%%%%%%%%%%
\begin{frame}[fragile]
    \frametitle{Diccionarios: algunos ejemplos}
    \begin{minted}{python}
    d = {}     # crea un diccionario vacío
    d = dict() # idem

    # un diccionario que contiene algunas constantes fundamentales
    cf = {'epsilon_0' : 8.854_187_8128e-12, 
          'hbar' : 6.626_070_15e-34/2/np.pi,
          'c' : 299_792_458, 'G' : 6.674_30e-11}

    # es indexable por sus llaves, aquí agregamos e
    d['e'] = 1.602_176_634e-19

    # indexamos con llaves como lo haríamos con una lista
    tiempo_de_planck = np.sqrt(cf['hbar'] *
                       cf['G'] / cf['c']**5) 
    \end{minted}
    \end{frame}

%%%%%%%%%%%%%%%%%%%%%%%%%%%%%%%%%%%%%%%%%%%%%%%%%%%%%%%%%%
\begin{frame}[fragile]
    \frametitle{Diccionarios: herramientas de iteración}
    \includegraphics[height=0.8cm, width=0.8cm]{figuras/sin.png}
    \begin{minted}{python}
    # iterando sobre las llaves
    for k in d.keys():
        # hago algo 

    # iterando sobre los pares de llave y valor
    for k,v in d.items():
        # tengo una tupla (k,v)
        # con cada llave y valor 
        # en cada iteración
    \end{minted}
    \end{frame}

%%%%%%%%%%%%%%%%%%%%%%%%%%%%%%%%%%%%%%%%%%%%%%%%%%%%%%%%%%%
\begin{frame}
\frametitle{Síntesis y recursos:}

\begin{itemize}
\item \href{https://numpy.org/doc/stable/}{Documentación de NumPy}
\item \href{https://numpy.org/doc/stable/reference/routines.math.html}{Funciones matemáticas en NumPy}

\end{itemize}
\end{frame}

\end{document}

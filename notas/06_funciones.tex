\documentclass{beamer}

\usepackage[utf8]{inputenc}
\usepackage[spanish]{babel}
\usepackage[outputdir=.build]{minted}
\usepackage{hyperref}
\usepackage{graphicx}

\hypersetup{
    colorlinks = true
}

\usetheme{Madrid}

%Information to be included in the title page:
\title{Taller de Física Computacional}
\subtitle{Funciones}
\author{Cristián G. Sánchez y Carlos J. Ruestes}
\date{2020}

\begin{document}

\frame{\titlepage}

%%%%%%%%%%%%%%%%%%%%%%%%%%%%%%%%%%%%%%%%%%%%%%%%%%%%%%%%%%
\begin{frame}[fragile]
\frametitle{Funciones}
\begin{block}{{\em Funciones}}
\begin{itemize}
\item Ya hemos visto algunas funciones que son parte de la biblioteca estándar como  \mintinline{python}{abs()} o \mintinline{python}{round(x,[n])}.
\item Las funciones en el marco de nuestro modelo de lo que es un programa son sub-programas.
\item Sirven para reducir el problema a porciones más pequeñas o simples.
\item Es una buena práctica ir armando una biblioteca de funciones para tenerlas a mano.
\item En Python las bibliotecas de funciones se guardan en \alert{módulos}.
\end{itemize}
\end{block}
\end{frame}

%%%%%%%%%%%%%%%%%%%%%%%%%%%%%%%%%%%%%%%%%%%%%%%%%%%%%%%%%%
\begin{frame}[fragile]
\frametitle{Funciones}
\begin{block}{{\em Funciones en Python}}
\begin{itemize}
\item Las funciones pueden tener un número fijo o variable de argumentos que puede ser cero.
\item Las funciones pueden tener argumentos opcionales (como el caso de round).
\item Las funciones pueden tener parámetros con nombre.
\item Las funciones pueden devolver un valor o no.
\item Las funciones pueden devolver más de un valor.
\item Las funciones pueden tener \alert{efectos colaterales}.
\item Las variables definidas dentro de una función sólo viven mientras la función se está ejecutando.
\end{itemize}
\end{block}
\end{frame}

%%%%%%%%%%%%%%%%%%%%%%%%%%%%%%%%%%%%%%%%%%%%%%%%%%%%%%%%%%
\begin{frame}[fragile]
\frametitle{Definición de funciones, sintaxis}
Usamos las palabras clave \mintinline{python}{def, return, None}, el delimitador \mintinline{python}{:} y los {\em tokens NUEVA LíNEA, INDENTACIÓN y DEDENTACIÓN}. Las siguientes son definiciones válidas de funciones
\begin{block}{}
\begin{minted}{python}
def identificador(p1,p2,p3=0):
   # pasan cosas
   # pasan más cosas
   return resultado 
 
def identificador(p1,p2):
   # pasan cosas
   # pasan más cosas
   # al no tener "return" esta función devuelve None
   
\end{minted}
\end{block}
\end{frame}

%%%%%%%%%%%%%%%%%%%%%%%%%%%%%%%%%%%%%%%%%%%%%%%%%%%%%%%%%%
\begin{frame}[fragile]
\frametitle{Definición de funciones, sintaxis}
Una función puede devolver más de un valor:
\begin{block}{}
\begin{minted}{python}
def identificador(p1,p2,p3=0):
   # pasan cosas
   # pasan más cosas
   return resultado1, resultado2    
\end{minted}
lo cual se invoca de la forma:
\begin{minted}{python}
a,b = identificador(p1,p2,p3)
\end{minted}
\end{block}
\end{frame}

%%%%%%%%%%%%%%%%%%%%%%%%%%%%%%%%%%%%%%%%%%%%%%%%%%%%%%%%%%
\begin{frame}[fragile]
   \frametitle{Alcance (Scope) y Efectos colaterales}
   \begin{itemize}
      \item Las variables definidas dentro de una función sólo viven mientras la función se está ejecutando.
      \item Las funciones pueden tener efectos colaterales, de hecho puede ser lo único que hagan.
      \item Una función puede utilizar variables variables globales \alert{(OJO)}
   \end{itemize}
    
   \begin{block}{}
   \begin{minted}{python}
   def identificador(a,b):
      c = 3.14 # c sólo vive mientras estoy aquí dentro
      c *= variable_global # Puedo usar una variable global
                           # si esté definida afuera
      print(c) # Este es un efecto colateral 
      resultado = a + b + c
      return resultado 
   \end{minted}
   \end{block}
   \end{frame}

%%%%%%%%%%%%%%%%%%%%%%%%%%%%%%%%%%%%%%%%%%%%%%%%%%%%%%%%%%

\begin{frame}[fragile]
   \frametitle{Definición de funciones, sintaxis}
   Las funciones pueden tener un número variable de argumentos tanto nombrados como sin nombre, para ello se utilizan las estructuras de datos {\em lista} y {\em diccionario} que veremos más adelante.
   \begin{block}{}
   \begin{minted}{python}
   def identificador(*argumentos):
      # pasan cosas
      # pasan más cosas
      # los argumentos entran por una "lista"
      return resultado 
    
   def identificador(**argumentos):
      # pasan cosas
      # pasan más cosas
      # los argumentos entran por un "diccionario"
      return resultado   
   \end{minted}
   \end{block}
   \end{frame}

%%%%%%%%%%%%%%%%%%%%%%%%%%%%%%%%%%%%%%%%%%%%%%%%%%%%%%%%%%%
\begin{frame}
\frametitle{Síntesis y recursos:}

\begin{itemize}
\item \href{https://docs.python.org/3/reference/index.html}{Manual de referencia de Python}
\item \href{https://docs.python.org/3/library/index.html}{Manual de la Librería estándar de Python}
\end{itemize}
\end{frame}

\end{document}

\documentclass{beamer}

\usepackage[utf8]{inputenc}
\usepackage[spanish]{babel}
\usepackage[outputdir=.build]{minted}
\usepackage{hyperref}
\usepackage{graphicx}

\hypersetup{
    colorlinks = true
}

\usetheme{Madrid}

%Information to be included in the title page:
\title{Taller de Física Computacional}
\subtitle{Estructuras de datos}
\author{Cristián G. Sánchez y Carlos J. Ruestes}
\date{2020}

\begin{document}

\frame{\titlepage}

%%%%%%%%%%%%%%%%%%%%%%%%%%%%%%%%%%%%%%%%%%%%%%%%%%%%%%%%%%
\begin{frame}[fragile]
    \frametitle{Estructuras de datos}
    Las estructuras de datos son la forma en que vemos y manipulamos esto:
    \begin{verbatim}
55 fb 89 2c b8 b7 84 26 #00000000 
b8 e0 15 04 31 45 e2 bb #00000001
44 a0 04 5a e6 d7 85 d2 #00000002
e1 6b 5d d3 a1 b9 f8 e6 #00000003
e3 6b 1d ff 39 06 c3 54 #00000004
2e 3d 4e 83 cf a4 3b 1d #00000005
54 94 45 42 81 6a b6 1d #00000006
3c 2b 75 bb 8c e9 2b d4 #00000007
    \end{verbatim}    
    de una forma más ``amigable''.
\end{frame}

%%%%%%%%%%%%%%%%%%%%%%%%%%%%%%%%%%%%%%%%%%%%%%%%%%%%%%%%%%
\begin{frame}[fragile]
    \frametitle{Estructuras de datos}
    \begin{block}{Datos en RAM}
        En la RAM los datos se representan en forma binaria, ordenados de acuerdo a una secuencia de {\em direcciones} que permiten al processador
        leerlos o escribirlos. 
    \end{block}
    \begin{block}{Estructuras de datos}
        Las estructuras de datos son formas de organización manejo y almacenamiento de los datos que permiten acceso y modificacion {\em eficientes}.
        Son abstracciones dentro del lenguaje que se representan de alguna manera en la RAM que \alert{en principio} no nos interesa.
    \end{block}
\end{frame}


%%%%%%%%%%%%%%%%%%%%%%%%%%%%%%%%%%%%%%%%%%%%%%%%%%%%%%%%%%%
\begin{frame}
    \frametitle{Estructuras de datos en Python}
    \includegraphics[height=0.8cm, width=0.8cm]{figuras/sin.png}
        Los tipos de datos (u objetos en general) en python pueden ser {\em mutables} o {\em inmutables}:
        \begin{block}{Objetos Inmutables}
            Son los objetos cuyo valor tienen un valor fijo, que no se puede cambiar una vez creados. Son inmutables, por ejemplo, los literales.
        \end{block}
        \begin{block}{Objetos Mutables}
            Son los objetos que pueden cambiar su valor pero mantienen su identificación (\mintinline{python}{id()})
        \end {block}        
\end{frame}

%%%%%%%%%%%%%%%%%%%%%%%%%%%%%%%%%%%%%%%%%%%%%%%%%%%%%%%%%%%
\begin{frame}
    \frametitle{Hash}
    \includegraphics[height=0.8cm, width=0.8cm]{figuras/sin.png}
    La inmutabilidad es un requisito para que un objeto sea {\em hasheable}.
    \begin{block}{Función {\em Hash}}
        Una función {\em hash} es un mapa (en principio unívoco) entre datos de tamaño arbitrario a valores de tamaño fijo. El valor puede ser utilizado como índice en una tabla ({\em hashtables}).
    \end{block}   
    \begin{itemize}
        \item La palabra {\em hash} tiene muchas traducciones literales que dependen del contexto. La de contexto culinario (picar y mezclar) es la que probablemente da origen a su uso. 
        No existe una palabra castellana exactamente equivalente y por tanto se utiliza sin traducir, algunos prefieren la palabra ``resumen''. 
    \end{itemize}   
\end{frame}

%%%%%%%%%%%%%%%%%%%%%%%%%%%%%%%%%%%%%%%%%%%%%%%%%%%%%%%%%%%
\begin{frame}
\frametitle{Estructuras de datos en Python}
    Aparte de las estructuras de datos simples que vimos hasta ahora (enteros, punto flotante, complejos, cadenas, etc.) Python implementa (entre otras) las siguientes
    estructuras de datos de forma nativa:
    \begin{block}{Secuencias}
        \begin{itemize}
            \item {\tt \bf tuple:} Secuencia inmutable usualmente utilizada para almacenar colecciones de elementos heterogéneos.
            \item {\tt \bf list:} Secuencia mutable usualmente utilizada para almacenar colecciones de elementos homogéneos.
            \item {\tt \bf range:} Secuencia inmutable de enteros ordenados, comunmente utilizada en bucles.       
        \end{itemize}
    \end{block}
\end{frame}

%%%%%%%%%%%%%%%%%%%%%%%%%%%%%%%%%%%%%%%%%%%%%%%%%%%%%%%%%%%%%
\begin{frame}
\frametitle{Estructuras de datos en Python}
\includegraphics[height=0.8cm, width=0.8cm]{figuras/sin.png}
    \begin{block}{Conjuntos}
        \begin{itemize}
            \item {\tt \bf set:} Colección no ordenada de objetos que deben ser {\em hasheables}, es mutable.
            \item {\tt \bf frozenset:} Un {\tt \bf set} inmutable.
        \end{itemize}
    \end{block}
    \begin{block}{Mapas}
        \begin{itemize}
            \item {\tt \bf dict} Una {\em hashtable} que mapea un objeto {\em hasheable} a otro objeto arbitrario. Los diccionarios son mutables. 
        \end{itemize}
    \end{block}

\end{frame}

%%%%%%%%%%%%%%%%%%%%%%%%%%%%%%%%%%%%%%%%%%%%%%%%%%%%%%%%%%%
\begin{frame}
    \frametitle{Síntesis y recursos:}

    Hay otras estructuras de datos implementadas en la biblioteca estándar, en particular en el módulo {\em collections}, para más detalles ver la sección Data Types del manual de la bilioteca estándar.
    
    \begin{itemize}
    \item \href{https://bit.ly/3nRJIJD}{Función Hash en Wikipedia}
    \item \href{https://www.amazon.com/Introduction-Algorithms-3rd-MIT-Press/dp/0262033844?tag=guru990f-20}{Introduction to Algorithms, Cormen {\em et al.}}
    \item \href{https://en.wikipedia.org/wiki/The_Art_of_Computer_Programming}{The Art of Computer Programming}
    \item \href{https://docs.python.org/3/reference/index.html}{Manual de referencia de Python}
    \item \href{https://docs.python.org/3/library/index.html}{Manual de la Librería estándar de Python}
    \end{itemize}
    \end{frame}

\end{document}

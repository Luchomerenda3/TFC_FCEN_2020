\documentclass{beamer}

\usepackage[utf8]{inputenc}
\usepackage[spanish]{babel}
\usepackage[outputdir=.build]{minted}
\usepackage{hyperref}
\usepackage{graphicx}
\usepackage{booktabs}

\hypersetup{
    colorlinks = true
}

\usetheme{Madrid}

%Information to be included in the title page:
\title{Taller de Física Computacional}
\subtitle{Secuencias: listas, tuplas y rangos}
\author{Cristián G. Sánchez y Carlos J. Ruestes}
\date{2020}

\begin{document}

\frame{\titlepage}


%%%%%%%%%%%%%%%%%%%%%%%%%%%%%%%%%%%%%%%%%%%%%%%%%%%%%%%%%%
\begin{frame}[fragile]
    \frametitle{Secuencias}
    Como vimos existen tres tipos de secuencias incorporadas en Python:
    \begin{block}{tuplas}
        Secuencias inmutables de elementos de distinto tipo $\dagger$.
    \end{block}
    \begin{block}{listas}
        Secuencias mutables de elementos del mismo tipo $\dagger$.
    \end{block}
    \begin{block}{rangos}
        Secuencias inmutables de enteros ordenados.
    \end{block}
    Se pueden definir otros objetos que se clasifican como secuencias si implementan una serie de métodos especiales.

    $\dagger$ El lenguaje no fuerza que los elementos cumplan esto.

\end{frame}


%%%%%%%%%%%%%%%%%%%%%%%%%%%%%%%%%%%%%%%%%%%%%%%%%%%%%%%%%%
\begin{frame}[fragile]
    \frametitle{Listas}
    \begin{block}{Listas}
    Se puede crear una lista en python de la siguientes formas:
        \begin{minted}[autogobble=true]{python}
        l_vacia = [] # crea una lista vacia
        l_vacia = list()

        # el argumento del constructor puede ser cualquier iterable
        l_importantes = list((3.14,2.71))

        l_enteros = [-3, -2, -1, 0, 1, 2, 3] # buen ejemplo

        # un ejemplo de la anarquia pytonica
        l_cosas = [3, 3.0, 'cadena', 1+3j, [-1,4,'mas cosas']]

        # algo que se parece a una matriz
        A = [[0.2, 0.3, 0.4], [0.5, 0.9, 3.4], [0.0, 1.0, 4.0]]
        \end{minted}
    \end{block}
    \end{frame}
    
%%%%%%%%%%%%%%%%%%%%%%%%%%%%%%%%%%%%%%%%%%%%%%%%%%%%%%%%%%
\begin{frame}[fragile]
    \frametitle{Tuplas}
    \begin{block}{Tuplas}
        \begin{itemize}
        \item Las tuplas en Python se usan fundamentalmente para devolver más de un valor en una función. Se utilizan en casos especiales para pasar a algunas 
    funciones parámetros que contienen otros parámetros. 
        \item Al ser inmutables su utilidad dentro de la programación científica (en nuestra opinión) es limitada, pero como son {\em hasheables}\/ pueden utilizarse
    como índices en los diccionarios (los veremos más adelante).
        \end{itemize}
    \end{block}
\end{frame}

%%%%%%%%%%%%%%%%%%%%%%%%%%%%%%%%%%%%%%%%%%%%%%%%%%%%%%%%%%
\begin{frame}[fragile]
    \frametitle{Tuplas}
    \begin{block}{Tuplas}
    Se pueden crear tuplas de la siguiente forma:
    \begin{minted}[autogobble=true]{python}
        # directamente
        planta = ('mimbulus','mimbletonia')
        alumno = ('Cedric','','Diggory','Hufflepuff')
        # la coma hace a la tupla
        enemigo = 'Tom','Marvolo','Riddle','Slythrin' 


        # a partir de una función que
        # devuelve varios valores
        def f(x):
            return x,x**2,x**3

        potencias_de_dos = f(2)
    \end{minted}
    \end{block}
    \end{frame}

%%%%%%%%%%%%%%%%%%%%%%%%%%%%%%%%%%%%%%%%%%%%%%%%%%%%%%%%%%
\begin{frame}[fragile]
    \frametitle{Rangos}
    \begin{block}{Rangos}
    Los rangos {\bf representan} (no contienen) una secuencia de enteros ordenados. Se generan usando el constructor de la clase, la función
    \mintinline{python}{range(start,stop,step)}, la cual se puede llamar con uno, dos o tres argumentos. Ejemplos:
    \begin{minted}[autogobble=true]{python}
        # Enteros de 0 a (stop-1)
        rango_1 = range(10) 

        # Enteros de start a (stop-1)
        rango2 = range(-10,11)

        # Enteros de start a (stop-1) de dos en dos (step) 
        rango2 = range(-10,11,2)
    \end{minted}
    \end{block}
\end{frame}

%%%%%%%%%%%%%%%%%%%%%%%%%%%%%%%%%%%%%%%%%%%%%%%%%%%%%%%%%%
\begin{frame}[fragile]
    \frametitle{Indexación}
    \begin{block}{Indexación}
    Todas las secuencias son indexables:
    \begin{itemize} 
        \item \mintinline{python}{seq[i]} es el i-ésimo elemento de la secuencia.
        \item \mintinline{python}{seq[0]} es el primer elemento de la secuencia.
        \item \mintinline{python}{seq[i:j]} es una {\em rebanada} de la secuencia, es la sub-secuencia desde
        el índice $i$ \alert{al $j-1$}.
        \item \mintinline{python}{seq[-1]} es el último elemento. Cualquier índice negativo
        sirve para indexar desde el final de la secuencia.
        \item Una lista de listas se idexa de la forma \mintinline{python}{seq[i][j]}
    \end{itemize}
    Si bien las secuencias son más generales podemos pensarlas (y usarlas) como un vector de $n$ componentes.
    \end{block}
    \end{frame}

%%%%%%%%%%%%%%%%%%%%%%%%%%%%%%%%%%%%%%%%%%%%%%%%%%%%%%%%%%
\begin{frame}[fragile]
    \frametitle{Operaciones implementadas para secuencias (mutables o inmutables)}
{\small
    \begin{table}[]
    \begin{tabular}{@{}ll@{}}
    \toprule
    \textbf{Operación}           & \textbf{Resultado}                                                                  \\ \midrule
    \mintinline{python}{x in s}                        & \mintinline{python}{True} si un elemento de $s$ es igual a $x$, sino \mintinline{python}{False}                                  \\
    \mintinline{python}{x not in s}                   & \mintinline{python}{False} si un elemento de $s$ es igual a $x$, sino \mintinline{python}{True}                                   \\ \midrule
    \mintinline{python}{s + t}                        & la concatenation de $s$ y $t$                                                     \\ \midrule
    \mintinline{python}{s * n} o \mintinline{python}{n * s}  & equivalente a sumar $s$ a si misma $n$ veces                                         \\
    \mintinline{python}{s[i]}                     & \textit{$i$ésimo elemento de $s$, \alert{el primero es el 0}}                                                 \\
    \mintinline{python}{s[i:j]}                   & \alert{rebanada} de $s$ de $i$ a $j$                                                           \\
    \mintinline{python}{s[i:j:k]}                 & \alert{rebanada} de $s$ de $i$ a $j$ con paso $k$                                               \\
    \mintinline{python}{len(s)}                       & largo de $s$                                                                      \\
    \mintinline{python}{min(s)}                       & el menor elemento de $s$                                                               \\
    \mintinline{python}{max(s)}                       & el mayor elemento de $s$                                                                \\
    \mintinline{python}{s.index(x, i[, j]])} & índice de la primera ocurrencia de $x$ en $s$  \\
                                 & (a o antes de $i$ y antes de $j$) \\
    \mintinline{python}{s.count(x)}                   & número total de ocurrencias de $x$ in $s$                                            \\ \bottomrule
    \end{tabular}
    \end{table}
}
\end{frame}

%%%%%%%%%%%%%%%%%%%%%%%%%%%%%%%%%%%%%%%%%%%%%%%%%%%%%%%%%%
\begin{frame}[fragile]
    \frametitle{Operaciones implementadas para secuencias mutables (listas)}
% Please add the following required packages to your document preamble:
% \usepackage{booktabs}
{\small
    \begin{table}[]
    \begin{tabular}{@{}ll@{}}
    \toprule
    \textbf{Operación}    & \textbf{Resultado}                                                                          \\ \midrule
    \mintinline{python}{s[i] = x}              & reemplaza el ítem $i$ de $s$ por $x$                                                             \\
    \mintinline{python}{s[i:j] = t}            & la rebanada de $s$ desde $i$ a $j$ se reemplaza \\ & por el contenido del iterable $t$                     \\ 
    \mintinline{python}{del s[i:j]}            & equivalente a \mintinline{python}{s[i:j] = []}                                                              \\ 
    \mintinline{python}{s[i:j:k] = t}          & los elementos de \mintinline{python}{s[i:j:k]} son reemplazados por los de $t$                                  \\
    \mintinline{python}{s *= n}                & actualiza $s$ con su contenido repetido $n$ veces                                             \\
    \end{tabular}
    \end{table}
}  

\end{frame}

%%%%%%%%%%%%%%%%%%%%%%%%%%%%%%%%%%%%%%%%%%%%%%%%%%%%%%%%%%
\begin{frame}[fragile]
    \frametitle{Operaciones implementadas para secuencias mutables (listas)}
% Please add the following required packages to your document preamble:
% \usepackage{booktabs}
{\small
    \begin{table}[]
    \begin{tabular}{@{}ll@{}}
    \toprule
    \textbf{Operation}    & \textbf{Result}                                                                          \\ \midrule
    \mintinline{python}{s.append(x)}           & agrega $x$ al final de la secuencia            \\
    \mintinline{python}{s.clear()}             & remueve todos los elementos de $s$                                          \\
    \mintinline{python}{s.copy()}              & crea una copia \alert{superficial} de $s$                                          \\
    \mintinline{python}{s.extend(t)} o \mintinline{python}{s += t} & extiende $s$ con el contenido de $t$  \\
    \mintinline{python}{s.insert(i, x)}        & inserta $x$ en $s$ en el índice $i$                  \\
    \mintinline{python}{s.pop([i])}            & devuelve y remueve de $s$ el elemento en $i$         \\ 
    \mintinline{python}{s.remove(x)}           & remueve el primer elemento de $s$ cuyo valor es $x$                                \\
    \mintinline{python}{s.reverse()}           & invierte el orden de los elementos de $s$ sin hacer una copia                                                        
    \end{tabular}
    \end{table}
}  

\end{frame}

%%%%%%%%%%%%%%%%%%%%%%%%%%%%%%%%%%%%%%%%%%%%%%%%%%%%%%%%%%%
\begin{frame}
\frametitle{Síntesis y recursos:}

\begin{itemize}
    \item \href{https://docs.python.org/3/reference/index.html}{Manual de referencia de Python}
    \item \href{https://docs.python.org/3/library/index.html}{Manual de la Librería estándar de Python}
\end{itemize}
\end{frame}

\end{document}

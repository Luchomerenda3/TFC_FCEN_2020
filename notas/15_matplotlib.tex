\documentclass{beamer}

\usepackage[utf8]{inputenc}
\usepackage[spanish]{babel}
\usepackage[outputdir=.build]{minted}
\usepackage{hyperref}
\usepackage{graphicx}

\hypersetup{
    colorlinks = true
}

\usetheme{Madrid}

%Information to be included in the title page:
\title{Taller de Física Computacional}
\subtitle{matplotlib}
\author{Cristián G. Sánchez y Carlos J. Ruestes}
\date{2020}

\begin{document}

\frame{\titlepage}


%%%%%%%%%%%%%%%%%%%%%%%%%%%%%%%%%%%%%%%%%%%%%%%%%%%%%%%%%%
\begin{frame}[fragile]
    \frametitle{Bibliotecas gráficas en Python}
    \begin{block}{{\em Bibliotecas gráficas en Python}}
        Las siguientes son las bibliotecas gráficas más utilizadas en Python
    \begin{itemize}
    \item \href{https://matplotlib.org}{Matplotlib}: La más ``simple'', porque seguramente alguien hizo antes algo parecido a lo que querés hacer y lo escribió en algún lado.
    \item \href{http://seaborn.pydata.org}{Seaborn}: Más ``bella'', matplotlib con colores más lindos, se usa mucho junto con Pandas (gráficos estadísticos)
    \item \href{http://ggplot.yhathq.com}{ggplot}: Basada en la ``gramática de los gráficos'', portada desde R, énfasis en estadística.
    \item \href{https://docs.bokeh.org/en/latest/}{Bokeh}: De los que hacen Anaconda, creada para graficos interactivos sobre información en tiempo real.
    \item \href{https://plotly.com}{Plotly}: Gráficos interactivos en ``tableros de control''.
    \end{itemize}
\end{block}    
\end{frame}

%%%%%%%%%%%%%%%%%%%%%%%%%%%%%%%%%%%%%%%%%%%%%%%%%%%%%%%%%%
\begin{frame}[fragile]
    \frametitle{matplotlib}
    \includegraphics[width=5cm]{figuras/matplotlib.png}
    \begin{block}{matplotlib}
        \begin{itemize}
            \item El estándar {\em de facto} para visualización en Python.
            \item Poderoso para gráficos 2D.
            \item Complicado (como todas las bibliotecas gráficas).
            \item Permite hacer diseño programático de gráficos (no solo matplotlib).
            \item La documentación es una pesadilla.
            \item El camino: Usar alguna ``receta'' existente y modificarla.
        \end{itemize}
    \end{block}
    \end{frame}

\end{document}

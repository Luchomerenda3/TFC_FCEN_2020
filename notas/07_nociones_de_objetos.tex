\documentclass{beamer}

\usepackage[utf8]{inputenc}
\usepackage[spanish]{babel}
\usepackage[outputdir=.build]{minted}
\usepackage{hyperref}
\usepackage{graphicx}
\usepackage{amsmath}

\hypersetup{
    colorlinks = true
}

\usetheme{Madrid}

%Information to be included in the title page:
\title{Taller de Física Computacional}
\subtitle{Nociones de Objetos}
\author{Cristián G. Sánchez y Carlos J. Ruestes}
\date{2020}

\begin{document}

\frame{\titlepage}

%%%%%%%%%%%%%%%%%%%%%%%%%%%%%%%%%%%%%%%%%%%%%%%%%%%%%%%%%%
\begin{frame}[fragile]
\frametitle{Objetos}
\begin{block}{{\em Objetos}}
Conviene tener aunque sea una noción de la forma en que los objetos funcionan en Python porque los encontraremos más temprano que tarde.
\end{block}
\begin{block}{{\em Objetos y Clases}}
\begin{itemize}
\item Una {\bf Clase} contiene las definición de las {\em propiedades}, los {\em métodos} y las {\em operaciones} entre miembros de una clase.
\item Un {\bf Objeto} es una instancia de una clase.
\end{itemize}
\end{block}
\end{frame}

%%%%%%%%%%%%%%%%%%%%%%%%%%%%%%%%%%%%%%%%%%%%%%%%%%%%%%%%%
\begin{frame}[fragile]
\frametitle{Objetos}
\begin{block}{{\em Objetos}}
\begin{itemize}
\item Los {\em Objetos} en Python son una abstracción para los {\em datos}. 
\item Todo {\em dato} en un programa Python está representado por objetos o relaciones entre objetos.
\item Cada objeto tiene una {\em identidad}, un {\em tipo} y un {\em valor}.
\item La identidad y el tipo de un objeto no pueden cambiar (ojo).
\item El valor de un objeto puede cambiar (dependiendo del tipo).
\end{itemize}
\end{block}
\end{frame}

%%%%%%%%%%%%%%%%%%%%%%%%%%%%%%%%%%%%%%%%%%%%%%%%%%%%%%%%%%
\begin{frame}[fragile]
\frametitle{Objetos}

\begin{block}{{\em Propiedades}}
Las propiedades se acceden con la notación  \mintinline{python}{objeto.propiedad}
\end{block}
\begin{block}{{\em Métodos}}
Los métodos se acceden con la notación  \mintinline{python}{objeto.función([parámetros])} (\dag)\\
En Python los métodos de un objeto no necesariamente operan sobre sus propiedades.
\end{block} 
(\dag) La notación \mintinline{python}{[parámetros]} indica que lo que está entre corchetes puede estar o no. 
\end{frame}

%%%%%%%%%%%%%%%%%%%%%%%%%%%%%%%%%%%%%%%%%%%%%%%%%%%%%%%%%%%
\begin{frame}
\frametitle{Síntesis y recursos:}

\begin{itemize}
\item \href{https://en.wikipedia.org/wiki/Object-oriented_programming}{Programación Orientada a Objetos en Wikipedia}
\item \href{https://docs.python.org/3/reference/index.html}{Manual de referencia de Python}
\item \href{https://docs.python.org/3/library/index.html}{Manual de la Librería estándar de Python}
\end{itemize}
\end{frame}

\end{document}
